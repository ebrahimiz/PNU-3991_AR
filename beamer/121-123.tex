\documentclass [8pt]{beamer}
\title{Mehran Soufi}
\author{}
\usepackage{xcolor}	
\usepackage{tikz}
\usepackage{textgreek}
\usetheme{Frankfurt}
\useoutertheme{infolines}
\usepackage{ragged2e}
\usepackage{amsmath}
\usepackage{amssymb}

\date{}
\begin{document}
\small
\begin{frame}
\justifying	
 \begin{itemize}
	\item 
	The research problem does not lend itself to well-defined systematic techniques; it can, however, acquire useful results from subjective judgments on a collective basis. 
	\item
	The research participants will be representative of diverse backgrounds, with respect to experience and expertise, and are geographically dispersed, making frequent group meetings virtually impossible due to time and cost.
	\item
	Related to the prior point that the participants will have diverse backgrounds), the experts may be uncompromising between opinions in a way that the communication process must be refereed and/or anonymity assured. 
	\item
	The heterogeneity of the research participants must be preserved to avoid domination by some experts.
\end{itemize}



A number of related research techniques have been developed that gather and condense opinions from groups of experts, which we refer to as consensus-generating techniques. The most well known of these is the Delphi Method, which has been used for over fifty years to resolve a wide variety of policy, forecasting, and decision-making problems. Consensus techniques have mostly been used in face-to-face conference modes; however, we are seeing increased interest in their use at a distance using a variety of communications technologies. The Net, with its capacity to support a variety of synchronous and asynchronous, as well as group and individual communications modes, is an ideal environment to support existing and experiment with new varieties of consensus data collection. In this chapter we overview the ways in which these techniques have been used and discuss ways the techniques can be adapted for Net use.

Generally, consensus techniques work by soliciting the opinion(s) of experts (usually in some sort of individual format) on an important issue or question. The researcher sometimes provides background materials or suggested information references that participants can consult to better inform their position. The range of the group responses, along with their individual opinions, are then returned to the participant. The participant is asked to defend, explain, or change their opinions so as to move the group to a single-best answer or consensus. This process may be repeated two or more times and hopefully as individual differences are reduced, a consensus develops. An important feature of the process is the facilitation and encouragement of individuals to share the rationale for their opinions-especially if these differ from the group mean. This sharing is facilitated through distribution of the written rationale or comments to the survey questions (using postal services, fax, email, or Web) or through a structured discussion during a face-to-face meeting or real-time distributed meeting or conference.
\end{frame}
\begin{frame}
\justifying	
The objective of the consensus process is to arrive at a single statement or answer that participants can agree on. Failing this unanimity, the process should clearly identify the nature and extent of opinion divergence. Generally consensus is sought; however, we are mindful of Mahatma Gandhi's observation that "honest disagreement is often a good sign of progress." A measurement of central tendency, the interquartile range (IQR), or the simple mean is usually used to determine the consensus of opinion for each of the questions posed to the group. Opposing opinions (those who fall outside
the IQR and median or bimodal distributions) are also noted and shared with the group to ensure that critical opposing opinions are not ignored. The process thereby dictates a conceptual communication structure that relates the opposing opinions to the data and objective of the research project. Opposing opinions are not considered to be antithetical with objectivity; rather, opposing opinions actually serve objectivity. This technique, then, may not lead to a convergence of opinions; bimodal distributions will always remain a possible outcome. However, the resulting outcome, irrespective of whether or not an opinion synthesis occurs, may be more valid than other methodologies because of the acknowledgment and accommodation of opposing opinions. Through the progress of successive iterations, the consensus process works to evolve an informed and well-thought-out answer to a difficult question. The question is often of such complexity or deals with future forecasts for which there is no way to calculate a single correct answer. As such, consensus techniques can serve to gather and articulate communal wisdom, as well as serve a cohesive and community function, bringing diverse opinions together and allowing individuals to work as an effective group.

The consensus process benefits if participants are able to view their opinions in relationship to those of the rest of the group and are given an opportunity and motivation to argue and defend their opinions. The resulting dialogue allows individuals (and the group) to alter and refine their opinions, hopefully leading to an informed and wise consensus of all participants. On the other hand, there is a danger that the consensus process will capture only collective ignorance. However, the selection of informed and motivated participants coupled with clear goals, objectives, and processes, usually results in consensus agreements that gather and expand the wisdom of all members.

Consensus-building theory has evolved into a series of techniques known as consensus research with three methodological variations or processes-Delphi Method, Nominal Groups Technique, and Consensus Development Conference (Murphy et al., 1998). The Net provides ways to not only expand, but also improve, both the effectiveness and efficiency of these traditional forms of consensus research.
\end{frame}
\begin{frame}
\justifying
\vspace{0.1cm}
{\large\textbf{ADVANTAGES OF CONSENSUS TECHNIQUES}}

Consensus techniques for e-research are related to focus groups and offer many of the same benefits and challenges. However, they provide more formal structure than focus group discussions. They are more deliberately focused on achieving a single best answer or statistically revealing the extent of disagreement than the open-ended and qualitative nature of most focus group research. Moreover, consensus groups provide a number of useful advantages for e-researchers seeking a means to gather knowledge from a dispersed group of experts.

\vspace{0.1cm}
{\large\textbf{Advantages of Traditional Consensus Techniques}}
\vspace{0.1cm}

\textbf{High-Quality and Informed Opinion.} Consensus groups are usually purposively selected so that the participants are informed, interested, and capable of providing highquality opinions. Participants in consensus groups draw first on their own experiences
and opinions, and then build on that knowledge by considering the opinions and expertise of others. This creates an environment for social cognition that is likely to produce better decisions than those made by individuals and to arrive at negotiated consensus from expert opinions. As Langford (1972) notes, the consensus data gathering technique endeavors to make "effective use of informed intuitive judgment; it is designed to combine individual judgments systematically and thus obtain a reasoned consensus" (p. 21).

\textbf{Safety in Numbers.} Consensus groups are less likely to arrive at or support incorrect answers or ineffective solutions because they are working with the collective expertise of a number of experts with a variety of experiences.

\textbf{Authority.} Group decisions are more likely to be taken seriously than those of any individual. In addition, specific consensus techniques have been shown to be more reliable and valid than other forms of opinion gathering and synthesis.

\textbf{Controlled Process.} Consensus techniques provide a set of procedures that tend to mitigate the negative impacts of group behavior, such as coercion, domination by certain individuals, or premature consensus seeking. A structured process can eliminate these kinds of group behavior.

\textbf{Support Communication among Individuals with Oplarized Views} Although consensus techniques may not always result in individuals coming to a unified position, they do create an environment in which polarized views can be democratically expressed and negotiated with equity. In some applications, anonymity is used to allow participants to freely state and argue their positions without threat of retaliation.

\textbf{Credibility.} Although these techniques are not without their technical critics, various mathematical techniques can be applied at each stage of the process to quantify individual and group opinions. The feedback of results to individuals allows participants to judge their opinions in relationship to the larger group. At the end of the process the extent of consensus can be accurately calculated and discussed.

\end{frame}
\begin{frame}
  \textbf{Accessibility.} Some types of consensus groups have evolved through the use of faceto-face meetings, and more recently these groups have been aided by the instant computation of the results of their decision using computers and software systems generally known as decision-making software. To overcome the time and space restrictions and costs related to face-to-face meetings, consensus techniques have also used postal mail or courier services to allow members to post and defend their reasons without meeting face-to-face. However, the inherent time delay, inconvenience, and cost of postal returns remain problematic.


\vspace{0.1cm}
{\large\textbf{Additional Advantages}}
\vspace{0.1cm}

To these advantages of traditional consensus techniques, a number of others can be added when the Net is used as the means of communication.
  
\end{frame}
\end{document}
